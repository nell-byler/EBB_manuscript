\chapter{Introduction}
\label{ch:intro}

%\begin{quote} 
% from here:
% http://solarnews.nso.edu/1999/19990902.html#SECTION00080000000000000000
%{\it ``If the sun didn't have a magnetic field, then it would be as boring a star as most astronomers think it is.''} [attributed to R.~B. Leighton]
%\end{quote}


The goal of this thesis is to XXX.



%%%%%%%%%%%%%%
\section{Outline of this Thesis}
The outline for the remainder of my thesis is as follows. In Chapter \ref{ch:model}, I present a new nebular emission model integrated within the Flexible Stellar Population Synthesis code that computes the line and continuum emission for complex stellar populations using the photoionization code \textsc{Cloudy}. The self-consistent coupling of the nebular emission to the matched ionizing spectrum produces emission line intensities that correctly scale with the stellar population as a function of age and metallicity.

Next, in Chapter \ref{ch:UV}, I talk about emission lines in the UV.

Then in Chapter \ref{ch:pAGB} I talk about low-ionization emission regions (LIERs).

Finally, in Chapter \ref{ch:summary} I provide a short discussion on the current and future context of my thesis research, and important directions this work will take in the next few years.
